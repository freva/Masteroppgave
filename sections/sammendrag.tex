%!TEX root = ../report.tex
\begin{otherlanguage}{norsk}

\section*{Sammendrag}
I de siste årene har mikroblogging på internet blitt en populær måte å utrykke egne tanker og følelser. Mikrobloggingstjenesten Twitter er ledende innenfor dette området, med over 320 millioner aktive brukere på verdensbasis. Den store mengden av mikroblogger som blir lagt ut via tjenesten hver dag gjør Twitter til en god datakilde for tekster med meningsytringer. Innenfor fagområdet sentimentanalyse, som går ut på å automatisk bestemme sentimentet i en tekst, har analyse av Twitter meldinger blitt populært de siste årene. Dette har ført til skapelsen av det nye fagområdet: Twitter sentimentanalyse. \\

I denne Masteroppgaven utforskes fagområdene leksikon-basert sentimentanalyse og automatisk generering av sentiment-leksikon. Basert på tidligere forskning innen fagområdene har både et system for automatisk generering av sentiment-leksikon samt et system for leksikon-basert sentimentanalyse blitt utviklet. \\

Ved å benytte vårt beste automatisk genererte leksikon som sentiment-leksikon i det leksikon-baserte sentimentanalyse systemet, oppnår vi gode resultater sammenlignet med maskinlæringsbaserte sentimentanalysesystemer. Når det gjelder kjøretidsytelse, utkonkurrerer systemet alle andre sammenlignede systemer, og beviser dermed sin evne til å fungere som en sanntids klassifikator av store mengder tweets på kort tid. I et sammenlignings\-eksperiment hvor vårt beste automatisk genererte sentiment leksikon sammenlignes med andre sentiment leksika, oppnår leksikonet vårt bedre resultater enn et manuelt annotert leksikon, noe som videre styrker posisjonen til \ac{pmi} metoden samt automatisk genererte--- overfor manuelt annoterte ---leksika. \\

I tillegg har vi oppdaget viktigheten av å spesialtilpasse klassifikatoren til vært individuelle sentiment leksika, samt at kvaliteten av sentiment leksikon laget med \ac{pmi} metoden er svært avhengig av kvaliteten på det benyttede annoterte datasettet.


\end{otherlanguage}

\glsresetall
